\enlang \enquotes
\fontfam[Latin Modern]
\report
\let\_firstnoindent=\relax
\activettchar`

%\font\mm=logo10 scaled 1139
%\def\MMTeX{{\mm MM}\-\TeX}
\def\MMTeX{MM\TeX}
\def\pdfTeX{pdf\TeX}
\def\Metapost{Metapost}
\def\TeXLive{\TeX\ Live}
\def\eTeX{\leavevmode\hbox{$\varepsilon$}-\TeX}

\tit \MMTeX

\author Michal Vlasák, 2020

\sec Introduction

\MMTeX\ is a minimal modern \TeX\ (\LuaTeX) distribution for Linux. It aims to
provide small base for Plain \LuaTeX\ and \OpTeX\ formats, but doesn't limit
their potential.

\MMTeX\ was created as a semestral work for the purposes of BI-TEX course at
FIT CTU in Prague. It's repository is hosted at
\url{https://github.com/vlasakm/mmtex/}.

\sec Motivation

I have been dissatisfied with \TeXLive. It is complicated and huge. The full
scheme installation, which is the default, takes $7\,075\,\hbox{MiB}$. Even the
minimal scheme which only includes Plain \TeX\ is 106\,MiB in size. There are
options to exclude source files and documentation which reduce size of full
scheme to 3\,899\,MiB, but that is still a lot.

As a monolith it also doesn't really fit very much into a Unix system. It's
default locations of `TEXMF` trees are not complaint to `hier(7)`. Instead,
everything is in an isolated tree.

I also wanted to explore how things really worked and what is really needed to
get Plain running, while not missing any of \LuaTeX's features.

\sec Target users

We target users who want to easily install a small ($\approx 31\,\hbox{MiB}$),
but very functional \TeX\ system. As \MMTeX\ includes \OpTeX\ it can even serve
as a replacement of \LaTeX\ and it's packages.

We also target use in Docker images or CI/CD scripts where the small size
matters too. Additionally being able to install \MMTeX\ from a Linux
distribution package manager makes it easy to use.

\sec \LuaTeX\ as the only \TeX\ engine

Historically there have been a few \TeX\ engines. Each brought it's advancement
over Knuth's original `tex`. The one which has had most development over the
last years is \LuaTeX. It's distinct aspect is that a Lua language interpreter
is baked into it. A lot of aspects of how \TeX\ works can now be influenced or
completly changed with Lua code. Another nice feature is integrated \Metapost\
in the form of `mplib`.

But for a \TeX\ outsider the main benefits are that it outputs PDF files, fully
supports Unicode and can use OpenType fonts. There aren't many reasons for not
using \LuaTeX\ for new documents. That is why \MMTeX\ includes only this
enginge.

\sec \OpTeX\ and Plain \LuaTeX\ formats

The first choice for a format was Plain \TeX\ or rather it's adaptation for
\LuaTeX. Plain has what most users would need anyway, and has been a base for
other formats. One of them is \OpTeX\ which is a \LuaTeX\ only format. It keeps
the simplicity of Plain \TeX\ while in a sense offers even more then \LaTeX\
--- where \LaTeX\ needs a package or external binary, \OpTeX\ has it built in.

As the dependencies of these two formats are mostly the same they are both
included in \MMTeX\ without much of an overhead.

\sec Language support

Previous \TeX\ engines had the limitation of being able to load hyphenation
patterns only at format creation time --- when running ini\TeX. \LuaTeX\ has no
such limitation and by using Lua it is possible to load hyphenation patterns at
any time.

Nowadays virtually all hyphenation patterns that have been used by \TeX\ users
are included in the `hyph-utf8` package. All patterns are provided in the form
usable by older \TeX\ enginges (which had to deal with 8-bit font encodings)
and the new plain text patterns directly loadable by \LuaTeX.

To save space and to not include hyphenation patterns of all languages in the
format, \TeXLive provides hyphenation patterns for each language in a different
package. Each package then registers itself using the \TeXLive \"execute
AddHyphen". An example for Czech language:

\begtt
execute AddHyphen \
	name=czech \
	lefthyphenmin=2 \
	righthyphenmin=3 \
	file=loadhyph-cs.tex \
	file_patterns=hyph-cs.pat.txt \
	file_exceptions=hyph-cs.hyp.txt
\endtt

This information is written to the `language.dat`, `language.def` and
`language.dat.lua` files. Only the last two are used by \eTeX\ language
mechanism, which is used by Plain \LuaTeX.

This is what the above `AddHyphen` directive adds to `language.def`:

\begtt
\addlanguage{czech}{loadhyph-cs.tex}{}{2}{3}
\endtt

and this is what gets added to `language.dat.lua`:

\begtt
	['czech'] = {
		loader = 'loadhyph-cs.tex',
		lefthyphenmin = 2,
		righthyphenmin = 3,
		synonyms = {  },
		patterns = 'hyph-cs.pat.txt',
		hyphenation = 'hyph-cs.hyp.txt',
	},
\endtt

Normally `language.def` is read by `etex.src` at format creation time and for
listed language registers them and loads their hyphenation patterns. That
enables their use later with `\uselanguage`.

As in \LuaTeX\ it is even discouraged to load patterns into format, the
mechanism is changed by `hyph-utf8`'s own `etex.src`. Instead of loading the
patterns on `\addlanguage` the language is only registred and the patterns are
loaded on the first `\uselanguage`. Both `\addlanguage` and `\uselangauge`
actually use `luatex-hyphen.lua`, which uses information in `language.dat.lua`
for handling synonyms and finding the names of pattern files (other fields in
`language.dat.lua` are unused).

\MMTeX\ includes all hyphenation patterns in the `txt` format --- they take up
3.2\ MiB, which is still reasonable. To enable the use of all of them a
`language.def` file is generated, which allows `etex.src` to register all
languages. A minimal `language.dat.lua` file is also generated and contains the
file names of pattern files of all supported languages.

In \OpTeX\ the situation is simpler as it already includes the information that
\eTeX\ expects in  `language.def`. However it still uses `luatex-hyphen.lua`
and `language.dat.lua`.

\sec Fonts

To use \LuaTeX\ to it's full potential it is better to use OpenType fonts.
These are the same fonts that are used by other programs and as such some of
them are already preinstalled on operating systems. And probably many more are
additionally installed by users or administrators. The idea is to let users use
the fonts they already have (or can get) on their system / TDS tree.

Traditionally Computer Modern family of fonts has been used as the default for
\TeX. There are OpenType adaptations of Computer Modern --- most notably Latin
Modern. But Latin Modern fonts are often already packaged for many systems. For
example there is `fonts-lmodern` on Debian based systems and `otf-latin-modern`
on Arch Linux. As \MMTeX\ aims to integrate well into these systems it doesn't
provide any OpenType fonts and instead expects the user to install fonts in the
standard ways. \MMTeX\ can easily find and use them.

\secc 8-bit fonts

Only 8-bit fonts can be preloaded into a \TeX\ format. Both \OpTeX\ and Plain
do it. To support this \MMTeX\ includes minimal set of Type 1 fonts and their
respective metric and encoding files. A `pdftex.map` file is needeed, as it is
used to map names of `.tfm` metric files, to font names and font files, with
optional reencodings. \pdfTeX and \LuaTeX\ only read one `pdftex.map` file, but
each font package usually provides one or more `.map` files. That is why an
aggregate `pdftex.map` is usually generated. As \MMTeX\ supports only limited
number of Type 1 fonts a minimal `pdftex.map` was created by hand. As this file
isn't supposed to be used by `dvips` it's directives are not included. This is
an example line from `pdftex.map`:

\begtt
cmr5 CMR5 <cmr5.pfb
\endtt

This line makes connection between cmr5.tfm metric file, cmr5.pfb Type 1 font
file and CMR5 font name. (CMR5 stands for Computer Modern Roman in 5 point
optical size).

A more complicated example:

\begtt
ec-lmr5 LMRoman5-Regular <lm-ec.enc <lmr5.pfb
\endtt

This is similiar as the previous case but additionally uses the encoding vector
stored in file `lm-ec.enc`. This is necessary because `lmr5.pfb` actually
contains many glyphs, while \TeX\ can use only 256 of them and expects the
order to correspond with `ec-lmr5.tfm` which contains metric information for
those selected 256 glyphs. In this particular case the Cork (\"EC") encoding is
used.

\secc OpenType fonts

In order to handle OpenType fonts some Lua code is needed. `luaotfload` is
included for this purpose as it is already used internally by \OpTeX. It can
also be used from Plain \LuaTeX, by using `\input luaotfload.sty`.

`luaotfload` is able to find all system fonts, because it reads `font-config`
configuration. This means that there is no need to set the `OSFONTDIR`
variable. Standard `TDS` directories for font files (namely
`$TEXMF/fonts/opentype` and `$TEXMF/fonts/truetype`) also work.

\sec `luamplib`

\MMTeX\ includes `luamplib` package which provides user level macros for using
`mplib`. `mplib` is \Metapost\ in form of a library built into \LuaTeX. To use
it: `\input luamplib.sty`.

\sec Finding files

\LuaTeX\ uses the `kpathsea` library for finding files. `kpathsea` uses path
specifications and variables similiar to the `PATH` environment variable. But
different variables are used for different file types. For example the variable
`TEXINPUTS` determines the location of files for `\input` (most likely `.tex`
files). These variables can be set from (in order of significance):

\begitems
* environment variables,
* `texmf.cnf` files,
* defaults set at compilation.
\enditems

For finding `texmf.cnf` files the same path searching mechanism is used, the
variable being `TEXMFCNF` (but it cannot be set from configuration file). All
found `texmf.cnf`'s are taken into account. Earlier assignments in
configuration files override later ones. That's why in \MMTeX\ we compile
`kpathsea` such that it first looks for `texmf.cnf` in
`$XDG_CONFIG_HOME/mmtex/texmf.cnf` (where `$XDG_CONFIG_HOME` defaults to
`~/.config`) for user configuration, then in `/etc/mmtex/texmf.cnf`, for system
wide configuration.

We also use the variable `TEXMFDOTDIR` (which defaults to `.`) as a primary
search path for every file type (even for `texmf.cnf` files!). This can be
useful for temporary overrides on the command line, and was inspired by
\TeXLive.

Other interesting variables are set to comply with \"\TeX\ Directory Structure"
or \"XDG Base Directory" standards. All roots of TDS trees are expected in the
variable `TEXMF`, which currently doesn't have default value, but can be set
arbitrarily. The user (or system administrator) can set it at any level they
need at the moment (in the environment, system/user/local configuration file).

\sec Building \MMTeX

\MMTeX\ is easily packagable as one of it's goals is to integrate well into
Unix systems. The implementation of \MMTeX\ is actually a `build` script which
contains instructions for building \MMTeX\ to directory. This means that
packagers can use this build script and some wrapper to easily create a package
for a Linux distrubution or any other system. `package-builder` script is
included which along with some metadata uses `build` and can create packages in
`.deb`, Arch Linux and `tar.gz` formats.

\secc Build instructions

To build `mmtex` package of requested types, run:

\begtt
./package-builder [--all] [--deb] [--arch] [--tar] mmtex..
\endtt

For example to do a full build of all package types, run:

\begtt
./package-builder --all mmtex
\endtt

\sec Installing \MMTeX

\secc Arch Linux

After obtaining `.pkg.tar.zst` file, you can install it by executing:

\begtt
pacman -U mmtex-<VERSION>-x86_64.pkg.tar.zst
\endtt

\secc Debian-based systems

After obtaining `.deb` file, you can install it by executing:

\begtt
dpkg -i mmtex_<VERSION>_amd64.deb
\endtt

\secc Any Linux

You can \"install" the tarball to `/usr/local` (or any other prefix) by
executing:

\begtt
tar -C /usr/local -xf mmtex-<VERSION>.tar.gz
\endtt

\sec Using \MMTeX

After installing \MMTeX\ you should now have two programs available: `optex`
and `luatex`. These binaries let you use the `luatex` engine with either
\OpTeX\ or (\LuaTeX) plain formats respectively. Beware that only a few very
basic fonts are included in the distribution and for any reasonable document
you should use OpenType fonts. Either system fonts which are found
automatically or fonts installed in TEXMF trees according to TDS.

You should point \MMTeX\ to your TEXMF trees by setting the `TEXMF` Kpathsea
variable. As usual with Kpathsea, sort these directories by priority and
separate them with colons. The paths in `TEXMF` {\em should not} contain
trailing slashes. You can set the variable either in the environment or in a
configuration file. For example to use files from your own `TEXMF` trees in
`$HOME/texmf`, and `/usr/share/texmf`, do:
%
\begtt
export TEXMF="$HOME/texmf:/usr/share/texmf"
\endtt
%
in the shell from which you invoke \TeX, or create a configuration file called
`texmf.cnf` containing:
%
\begtt
TEXMF = $HOME/texmf:/usr/share/texmf
\endtt

You can put the configuration file in either the current directory,
`$XDG_CONFIG_HOME/mmtex` (where `$XDG_CONFIG_HOME` is usually `~/.config`), or
`/etc/mmtex`.

\secc \OpTeX

For using \OpTeX\ you should check its documentation at
\url{http://petr.olsak.net/ftp/olsak/optex/optex-doc.pdf}.

\secc Plain \LuaTeX

You should set your language with `\uselanguage{<LANGUAGE_NAME>}`. This is
neccessary for languages other than American English to ensure that hyphenation
works correctly. For list of available languages see the file `language.dat.lua`, which is located at
`$PREFIX/share/mmtex/tex/mmtex`, where `$PREFIX` is the prefix
for package installation (probably `/usr` if you installed \MMTeX\ with your package
manager, or maybe `/usr/local` if you installed the `.tar.gz` package).

You should use `luaotfload` for loading OpenType fonts. First you need to
`\input` it, and then you can define your fonts. For example:

\begtt
\input luaotfload.sty

\font\mainfont="Fira Sans Light" at 12pt
\mainfont

Hello there
\bye
\endtt

Note that plain \TeX's preloaded fonts (like `\tenrm`) will be still used for a lot of things,
if you don't want that, you need to redefine them manually.

\bye
